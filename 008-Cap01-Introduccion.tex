\chapter{Introducci'on}
\label{capitulouno}

El dise\~no de interfaces en el 'area del desarrollo de software es una actividad muy importante, la misma es realizada por los diferentes equipos de trabajo y comprende el trabajo conjunto de un grupo de personas, lo que implica colaboraci'on y coordinaci'on al momento de realizarse.

\medskip

La participaci'on de un grupo de personas puede representar una dificultad a la hora de dise\~nar una interfaz propiamente dicha, puesto que los miembros del equipo deben generalmente hallarse geogr'aficamente en el mismo lugar y al mismo tiempo para realizar alg'un aporte inmediato, o esperar a la retroalimentacion una vez terminado el dise\~no. La mayor parte de las herramientas que ayudan en el dise\~no de interfaces son del tipo mono usuario, esto significa que solo un usuario puede usarlas a la vez.

\medskip

El proyecto pretende realizar un sistema colaborativo para coadyuvar en la coordinación de dicha actividad a partir de una herramienta ya existente, con ello se pretende facilitar la participaci'on de los integrantes del equipo permiti'endoles dise\~nar una interfaz de manera conjunta usando algunos principios de CSCW para la adaptacion de un sistema mono usuario en un sistema multiusuario.


\section{Antecedentes}

La interfaz es el medio por el cual un usuario puede interactuar con un sistema. Por ello el dise\~no de la interfaz tiende a ser fundamental dentro del desarrollo de un proyecto.

\medskip
 
Al ser esta una actividad importante se involucra una cantidad considerable de personas, como el equipo desarroll'o, el usuario para el cual est'a pensado el proyecto y algunos otros que se consideren importantes para esta actividad.

Los equipos o grupos de trabajo al involucrar cantidades determinadas de personas requieren herramientas y t'ecnicas para hacer sus tareas de la mejor forma posible y con un alto nivel de colaboraci'on.

\medskip
Se han encontrado trabajos similares como: El Sistema de Dise\~no de interfaces ``pencil'' \cite{pencil2010}
  
``\textit{Sometimes you need to see through walls – a field study of application programming interface}''. ``A veces se necesita ver a trav'ez de las paredes - un campo de estudio de aplicaciones para programar interfaces'' \cite{de2004sometimes}

\section{Definic'ion del problema}

Gran parte de los equipos de desarrollo se re'unen para tratar el dise\~no de la interfaz a ser desarrollada, lo cual conlleva a que todos los miembros del equipo se encuentren en un lugar y en un mismo tiempo lo que puede presentar un grave problema al momento de coordinar el trabajo. 

\medskip

Los sistemas de apoyo al dise\~no de interfaces en su mayor'ia son del tipo mono-usuario, lo que quiere decir que pueden ser usados por un usuario en un determinado periodo de tiempo, esto representa una limitante al momento de realizar trabajo en equipo.

Al momento de realizar el dise\~no de la interfaz los miembros del equipo pueden brindar una amplia gama de ideas, y el llevar un registro de las propuestas o los cambios realizados se torna un poco dificultoso.

\medskip

Por  lo expresado anteriormente se define el problema como:
Inadecuada coordinaci'on en el equipo para el dise\~no de interfaces lo que ocasiona dificultad en la participaci'on de los miembros del equipo de trabajo.


\section{Objetivos}

\subsection{Objetivo General}

Apoyar la coordinaci'on del equipo en el dise\~no de interfaces mediante un Sistema Colaborativo facilitando de este modo la participaci'on de los integrantes del equipo.

\subsection{Objetivos espec'ificos}

\begin{enumerate}
	\item Brindar la posibilidad de trabajo multiusuario coadyuvando al trabajo en equipo.
	\item Facilitar la retroalimentaci'on del trabajo realizado mediante un historial de acciones relevantes realizadas. 
	\item Brindar una alternativa de trabajar en un mismo proyecto desde diferentes ubicaciones.
	\item Facilitar la retroalimentaci'on de las propuestas brindadas por los miembros del equipo de trabajo.
\end{enumerate}

\section{Justificaci'on}

El proyecto se realizara tras advertir la necesidad de los equipos de desarrollo de software por fomentar la participaci'on de la mayor cantidad de personas en el dise\~no de las interfaces de un proyecto.

Se usara principios de Middleware pues es muy 'util para permitir el funcionamiento de aplicaciones distribuidas sobre plataformas heterog'eneas y homogeneas.

Se pretende tambi'en demostrar los principios de adaptacion de un sistema mono usuario a un sistema multiusuario de manera sencilla. 

\section{Alcance}

Con el proyecto brindar una herramienta capaz de permitir el trabajo de dise\~no de interfaces de usuario con la participacion de un equipo de trabajo sin la necesidad de forzar a los mismos a usar una herramienta diferente a la que estan acostumbrados.

\section{Descripci'on del contenido}

El presente proyecto fue realizado segun el flujo de trabajo presentado en las herramientas de la metodolog'ia AMENITIES
